
% Default to the notebook output style

    


% Inherit from the specified cell style.




    
\documentclass{article}

    
    
    \usepackage{graphicx} % Used to insert images
    \usepackage{adjustbox} % Used to constrain images to a maximum size 
    \usepackage{color} % Allow colors to be defined
    \usepackage{enumerate} % Needed for markdown enumerations to work
    \usepackage{geometry} % Used to adjust the document margins
    \usepackage{amsmath} % Equations
    \usepackage{amssymb} % Equations
    \usepackage[mathletters]{ucs} % Extended unicode (utf-8) support
    \usepackage[utf8x]{inputenc} % Allow utf-8 characters in the tex document
    \usepackage{fancyvrb} % verbatim replacement that allows latex
    \usepackage{grffile} % extends the file name processing of package graphics 
                         % to support a larger range 
    % The hyperref package gives us a pdf with properly built
    % internal navigation ('pdf bookmarks' for the table of contents,
    % internal cross-reference links, web links for URLs, etc.)
    \usepackage{hyperref}
    \usepackage{longtable} % longtable support required by pandoc >1.10
    \usepackage{booktabs}  % table support for pandoc > 1.12.2
    

    
    
    \definecolor{orange}{cmyk}{0,0.4,0.8,0.2}
    \definecolor{darkorange}{rgb}{.71,0.21,0.01}
    \definecolor{darkgreen}{rgb}{.12,.54,.11}
    \definecolor{myteal}{rgb}{.26, .44, .56}
    \definecolor{gray}{gray}{0.45}
    \definecolor{lightgray}{gray}{.95}
    \definecolor{mediumgray}{gray}{.8}
    \definecolor{inputbackground}{rgb}{.95, .95, .85}
    \definecolor{outputbackground}{rgb}{.95, .95, .95}
    \definecolor{traceback}{rgb}{1, .95, .95}
    % ansi colors
    \definecolor{red}{rgb}{.6,0,0}
    \definecolor{green}{rgb}{0,.65,0}
    \definecolor{brown}{rgb}{0.6,0.6,0}
    \definecolor{blue}{rgb}{0,.145,.698}
    \definecolor{purple}{rgb}{.698,.145,.698}
    \definecolor{cyan}{rgb}{0,.698,.698}
    \definecolor{lightgray}{gray}{0.5}
    
    % bright ansi colors
    \definecolor{darkgray}{gray}{0.25}
    \definecolor{lightred}{rgb}{1.0,0.39,0.28}
    \definecolor{lightgreen}{rgb}{0.48,0.99,0.0}
    \definecolor{lightblue}{rgb}{0.53,0.81,0.92}
    \definecolor{lightpurple}{rgb}{0.87,0.63,0.87}
    \definecolor{lightcyan}{rgb}{0.5,1.0,0.83}
    
    % commands and environments needed by pandoc snippets
    % extracted from the output of `pandoc -s`
    \DefineVerbatimEnvironment{Highlighting}{Verbatim}{commandchars=\\\{\}}
    % Add ',fontsize=\small' for more characters per line
    \newenvironment{Shaded}{}{}
    \newcommand{\KeywordTok}[1]{\textcolor[rgb]{0.00,0.44,0.13}{\textbf{{#1}}}}
    \newcommand{\DataTypeTok}[1]{\textcolor[rgb]{0.56,0.13,0.00}{{#1}}}
    \newcommand{\DecValTok}[1]{\textcolor[rgb]{0.25,0.63,0.44}{{#1}}}
    \newcommand{\BaseNTok}[1]{\textcolor[rgb]{0.25,0.63,0.44}{{#1}}}
    \newcommand{\FloatTok}[1]{\textcolor[rgb]{0.25,0.63,0.44}{{#1}}}
    \newcommand{\CharTok}[1]{\textcolor[rgb]{0.25,0.44,0.63}{{#1}}}
    \newcommand{\StringTok}[1]{\textcolor[rgb]{0.25,0.44,0.63}{{#1}}}
    \newcommand{\CommentTok}[1]{\textcolor[rgb]{0.38,0.63,0.69}{\textit{{#1}}}}
    \newcommand{\OtherTok}[1]{\textcolor[rgb]{0.00,0.44,0.13}{{#1}}}
    \newcommand{\AlertTok}[1]{\textcolor[rgb]{1.00,0.00,0.00}{\textbf{{#1}}}}
    \newcommand{\FunctionTok}[1]{\textcolor[rgb]{0.02,0.16,0.49}{{#1}}}
    \newcommand{\RegionMarkerTok}[1]{{#1}}
    \newcommand{\ErrorTok}[1]{\textcolor[rgb]{1.00,0.00,0.00}{\textbf{{#1}}}}
    \newcommand{\NormalTok}[1]{{#1}}
    
    % Define a nice break command that doesn't care if a line doesn't already
    % exist.
    \def\br{\hspace*{\fill} \\* }
    % Math Jax compatability definitions
    \def\gt{>}
    \def\lt{<}
    % Document parameters
    
\title{Nonlinear Dynamics}

    
    
\author{Joshua Wall}

    

    % Pygments definitions
    
\makeatletter
\def\PY@reset{\let\PY@it=\relax \let\PY@bf=\relax%
    \let\PY@ul=\relax \let\PY@tc=\relax%
    \let\PY@bc=\relax \let\PY@ff=\relax}
\def\PY@tok#1{\csname PY@tok@#1\endcsname}
\def\PY@toks#1+{\ifx\relax#1\empty\else%
    \PY@tok{#1}\expandafter\PY@toks\fi}
\def\PY@do#1{\PY@bc{\PY@tc{\PY@ul{%
    \PY@it{\PY@bf{\PY@ff{#1}}}}}}}
\def\PY#1#2{\PY@reset\PY@toks#1+\relax+\PY@do{#2}}

\expandafter\def\csname PY@tok@gd\endcsname{\def\PY@tc##1{\textcolor[rgb]{0.63,0.00,0.00}{##1}}}
\expandafter\def\csname PY@tok@gu\endcsname{\let\PY@bf=\textbf\def\PY@tc##1{\textcolor[rgb]{0.50,0.00,0.50}{##1}}}
\expandafter\def\csname PY@tok@gt\endcsname{\def\PY@tc##1{\textcolor[rgb]{0.00,0.27,0.87}{##1}}}
\expandafter\def\csname PY@tok@gs\endcsname{\let\PY@bf=\textbf}
\expandafter\def\csname PY@tok@gr\endcsname{\def\PY@tc##1{\textcolor[rgb]{1.00,0.00,0.00}{##1}}}
\expandafter\def\csname PY@tok@cm\endcsname{\let\PY@it=\textit\def\PY@tc##1{\textcolor[rgb]{0.25,0.50,0.50}{##1}}}
\expandafter\def\csname PY@tok@vg\endcsname{\def\PY@tc##1{\textcolor[rgb]{0.10,0.09,0.49}{##1}}}
\expandafter\def\csname PY@tok@m\endcsname{\def\PY@tc##1{\textcolor[rgb]{0.40,0.40,0.40}{##1}}}
\expandafter\def\csname PY@tok@mh\endcsname{\def\PY@tc##1{\textcolor[rgb]{0.40,0.40,0.40}{##1}}}
\expandafter\def\csname PY@tok@go\endcsname{\def\PY@tc##1{\textcolor[rgb]{0.53,0.53,0.53}{##1}}}
\expandafter\def\csname PY@tok@ge\endcsname{\let\PY@it=\textit}
\expandafter\def\csname PY@tok@vc\endcsname{\def\PY@tc##1{\textcolor[rgb]{0.10,0.09,0.49}{##1}}}
\expandafter\def\csname PY@tok@il\endcsname{\def\PY@tc##1{\textcolor[rgb]{0.40,0.40,0.40}{##1}}}
\expandafter\def\csname PY@tok@cs\endcsname{\let\PY@it=\textit\def\PY@tc##1{\textcolor[rgb]{0.25,0.50,0.50}{##1}}}
\expandafter\def\csname PY@tok@cp\endcsname{\def\PY@tc##1{\textcolor[rgb]{0.74,0.48,0.00}{##1}}}
\expandafter\def\csname PY@tok@gi\endcsname{\def\PY@tc##1{\textcolor[rgb]{0.00,0.63,0.00}{##1}}}
\expandafter\def\csname PY@tok@gh\endcsname{\let\PY@bf=\textbf\def\PY@tc##1{\textcolor[rgb]{0.00,0.00,0.50}{##1}}}
\expandafter\def\csname PY@tok@ni\endcsname{\let\PY@bf=\textbf\def\PY@tc##1{\textcolor[rgb]{0.60,0.60,0.60}{##1}}}
\expandafter\def\csname PY@tok@nl\endcsname{\def\PY@tc##1{\textcolor[rgb]{0.63,0.63,0.00}{##1}}}
\expandafter\def\csname PY@tok@nn\endcsname{\let\PY@bf=\textbf\def\PY@tc##1{\textcolor[rgb]{0.00,0.00,1.00}{##1}}}
\expandafter\def\csname PY@tok@no\endcsname{\def\PY@tc##1{\textcolor[rgb]{0.53,0.00,0.00}{##1}}}
\expandafter\def\csname PY@tok@na\endcsname{\def\PY@tc##1{\textcolor[rgb]{0.49,0.56,0.16}{##1}}}
\expandafter\def\csname PY@tok@nb\endcsname{\def\PY@tc##1{\textcolor[rgb]{0.00,0.50,0.00}{##1}}}
\expandafter\def\csname PY@tok@nc\endcsname{\let\PY@bf=\textbf\def\PY@tc##1{\textcolor[rgb]{0.00,0.00,1.00}{##1}}}
\expandafter\def\csname PY@tok@nd\endcsname{\def\PY@tc##1{\textcolor[rgb]{0.67,0.13,1.00}{##1}}}
\expandafter\def\csname PY@tok@ne\endcsname{\let\PY@bf=\textbf\def\PY@tc##1{\textcolor[rgb]{0.82,0.25,0.23}{##1}}}
\expandafter\def\csname PY@tok@nf\endcsname{\def\PY@tc##1{\textcolor[rgb]{0.00,0.00,1.00}{##1}}}
\expandafter\def\csname PY@tok@si\endcsname{\let\PY@bf=\textbf\def\PY@tc##1{\textcolor[rgb]{0.73,0.40,0.53}{##1}}}
\expandafter\def\csname PY@tok@s2\endcsname{\def\PY@tc##1{\textcolor[rgb]{0.73,0.13,0.13}{##1}}}
\expandafter\def\csname PY@tok@vi\endcsname{\def\PY@tc##1{\textcolor[rgb]{0.10,0.09,0.49}{##1}}}
\expandafter\def\csname PY@tok@nt\endcsname{\let\PY@bf=\textbf\def\PY@tc##1{\textcolor[rgb]{0.00,0.50,0.00}{##1}}}
\expandafter\def\csname PY@tok@nv\endcsname{\def\PY@tc##1{\textcolor[rgb]{0.10,0.09,0.49}{##1}}}
\expandafter\def\csname PY@tok@s1\endcsname{\def\PY@tc##1{\textcolor[rgb]{0.73,0.13,0.13}{##1}}}
\expandafter\def\csname PY@tok@sh\endcsname{\def\PY@tc##1{\textcolor[rgb]{0.73,0.13,0.13}{##1}}}
\expandafter\def\csname PY@tok@sc\endcsname{\def\PY@tc##1{\textcolor[rgb]{0.73,0.13,0.13}{##1}}}
\expandafter\def\csname PY@tok@sx\endcsname{\def\PY@tc##1{\textcolor[rgb]{0.00,0.50,0.00}{##1}}}
\expandafter\def\csname PY@tok@bp\endcsname{\def\PY@tc##1{\textcolor[rgb]{0.00,0.50,0.00}{##1}}}
\expandafter\def\csname PY@tok@c1\endcsname{\let\PY@it=\textit\def\PY@tc##1{\textcolor[rgb]{0.25,0.50,0.50}{##1}}}
\expandafter\def\csname PY@tok@kc\endcsname{\let\PY@bf=\textbf\def\PY@tc##1{\textcolor[rgb]{0.00,0.50,0.00}{##1}}}
\expandafter\def\csname PY@tok@c\endcsname{\let\PY@it=\textit\def\PY@tc##1{\textcolor[rgb]{0.25,0.50,0.50}{##1}}}
\expandafter\def\csname PY@tok@mf\endcsname{\def\PY@tc##1{\textcolor[rgb]{0.40,0.40,0.40}{##1}}}
\expandafter\def\csname PY@tok@err\endcsname{\def\PY@bc##1{\setlength{\fboxsep}{0pt}\fcolorbox[rgb]{1.00,0.00,0.00}{1,1,1}{\strut ##1}}}
\expandafter\def\csname PY@tok@kd\endcsname{\let\PY@bf=\textbf\def\PY@tc##1{\textcolor[rgb]{0.00,0.50,0.00}{##1}}}
\expandafter\def\csname PY@tok@ss\endcsname{\def\PY@tc##1{\textcolor[rgb]{0.10,0.09,0.49}{##1}}}
\expandafter\def\csname PY@tok@sr\endcsname{\def\PY@tc##1{\textcolor[rgb]{0.73,0.40,0.53}{##1}}}
\expandafter\def\csname PY@tok@mo\endcsname{\def\PY@tc##1{\textcolor[rgb]{0.40,0.40,0.40}{##1}}}
\expandafter\def\csname PY@tok@kn\endcsname{\let\PY@bf=\textbf\def\PY@tc##1{\textcolor[rgb]{0.00,0.50,0.00}{##1}}}
\expandafter\def\csname PY@tok@mi\endcsname{\def\PY@tc##1{\textcolor[rgb]{0.40,0.40,0.40}{##1}}}
\expandafter\def\csname PY@tok@gp\endcsname{\let\PY@bf=\textbf\def\PY@tc##1{\textcolor[rgb]{0.00,0.00,0.50}{##1}}}
\expandafter\def\csname PY@tok@o\endcsname{\def\PY@tc##1{\textcolor[rgb]{0.40,0.40,0.40}{##1}}}
\expandafter\def\csname PY@tok@kr\endcsname{\let\PY@bf=\textbf\def\PY@tc##1{\textcolor[rgb]{0.00,0.50,0.00}{##1}}}
\expandafter\def\csname PY@tok@s\endcsname{\def\PY@tc##1{\textcolor[rgb]{0.73,0.13,0.13}{##1}}}
\expandafter\def\csname PY@tok@kp\endcsname{\def\PY@tc##1{\textcolor[rgb]{0.00,0.50,0.00}{##1}}}
\expandafter\def\csname PY@tok@w\endcsname{\def\PY@tc##1{\textcolor[rgb]{0.73,0.73,0.73}{##1}}}
\expandafter\def\csname PY@tok@kt\endcsname{\def\PY@tc##1{\textcolor[rgb]{0.69,0.00,0.25}{##1}}}
\expandafter\def\csname PY@tok@ow\endcsname{\let\PY@bf=\textbf\def\PY@tc##1{\textcolor[rgb]{0.67,0.13,1.00}{##1}}}
\expandafter\def\csname PY@tok@sb\endcsname{\def\PY@tc##1{\textcolor[rgb]{0.73,0.13,0.13}{##1}}}
\expandafter\def\csname PY@tok@k\endcsname{\let\PY@bf=\textbf\def\PY@tc##1{\textcolor[rgb]{0.00,0.50,0.00}{##1}}}
\expandafter\def\csname PY@tok@se\endcsname{\let\PY@bf=\textbf\def\PY@tc##1{\textcolor[rgb]{0.73,0.40,0.13}{##1}}}
\expandafter\def\csname PY@tok@sd\endcsname{\let\PY@it=\textit\def\PY@tc##1{\textcolor[rgb]{0.73,0.13,0.13}{##1}}}

\def\PYZbs{\char`\\}
\def\PYZus{\char`\_}
\def\PYZob{\char`\{}
\def\PYZcb{\char`\}}
\def\PYZca{\char`\^}
\def\PYZam{\char`\&}
\def\PYZlt{\char`\<}
\def\PYZgt{\char`\>}
\def\PYZsh{\char`\#}
\def\PYZpc{\char`\%}
\def\PYZdl{\char`\$}
\def\PYZhy{\char`\-}
\def\PYZsq{\char`\'}
\def\PYZdq{\char`\"}
\def\PYZti{\char`\~}
% for compatibility with earlier versions
\def\PYZat{@}
\def\PYZlb{[}
\def\PYZrb{]}
\makeatother


    % Exact colors from NB
    \definecolor{incolor}{rgb}{0.0, 0.0, 0.5}
    \definecolor{outcolor}{rgb}{0.545, 0.0, 0.0}



    
    % Prevent overflowing lines due to hard-to-break entities
    \sloppy 
    % Setup hyperref package
    \hypersetup{
      breaklinks=true,  % so long urls are correctly broken across lines
      colorlinks=true,
      urlcolor=blue,
      linkcolor=darkorange,
      citecolor=darkgreen,
      }
    % Slightly bigger margins than the latex defaults
    
    \geometry{verbose,tmargin=1in,bmargin=1in,lmargin=1in,rmargin=1in}
    
    

    \begin{document}
    
    
    \maketitle
    
    

    

    \section{Nonlinear Dynamics Homework \#1}



    \subparagraph{Problem \#1: Construct 10,000 random numbers and bin them in intervals
of length 0.01. Plot the histogram of this.}


    Okay, so Python has a nice in that the histogram command bins for us.
But first I'll make 10,000 random numbers, using the best random number
generator in Python. It makes random numbers based on the system CPU
that are suitable for cryptography (see
https://docs.python.org/2/library/os.html\#os.urandom for details). Note
that the generator defaults to the range \([0,1)\). The best I can do to
make it \((0,1)\) is to use my machince precision to get the edges
right. Hopefully that's okay.

    \begin{Verbatim}[commandchars=\\\{\}]
{\color{incolor}In [{\color{incolor}1}]:} \PY{k+kn}{import} \PY{n+nn}{random}
        \PY{k+kn}{import} \PY{n+nn}{numpy} \PY{k+kn}{as} \PY{n+nn}{np}
        \PY{k+kn}{import} \PY{n+nn}{matplotlib}
        \PY{k+kn}{import} \PY{n+nn}{matplotlib.pyplot} \PY{k+kn}{as} \PY{n+nn}{plt}
        \PY{k+kn}{import} \PY{n+nn}{sys}
        \PY{o}{\PYZpc{}}\PY{k}{matplotlib} \PY{n}{inline}
\end{Verbatim}

    \begin{Verbatim}[commandchars=\\\{\}]
{\color{incolor}In [{\color{incolor}2}]:} \PY{n}{rands} \PY{o}{=} \PY{n}{np}\PY{o}{.}\PY{n}{zeros}\PY{p}{(}\PY{l+m+mi}{10000}\PY{p}{)}
        
        \PY{n}{foo} \PY{o}{=} \PY{n}{random}\PY{o}{.}\PY{n}{SystemRandom}\PY{p}{(}\PY{p}{)}
        
        \PY{k}{for} \PY{n}{i} \PY{o+ow}{in} \PY{n+nb}{range}\PY{p}{(}\PY{l+m+mi}{10000}\PY{p}{)}\PY{p}{:}
            
            \PY{n}{rands}\PY{p}{[}\PY{n}{i}\PY{p}{]} \PY{o}{=} \PY{n}{foo}\PY{o}{.}\PY{n}{uniform}\PY{p}{(}\PY{l+m+mi}{0}\PY{o}{+}\PY{n}{sys}\PY{o}{.}\PY{n}{float\PYZus{}info}\PY{o}{.}\PY{n}{epsilon}\PY{p}{,} \PY{l+m+mf}{1.0} \PY{o}{\PYZhy{}} \PY{n}{sys}\PY{o}{.}\PY{n}{float\PYZus{}info}\PY{o}{.}\PY{n}{epsilon}\PY{p}{)}
            
        \PY{k}{print} \PY{n}{rands}
\end{Verbatim}

    \begin{Verbatim}[commandchars=\\\{\}]
[ 0.6090817   0.02900779  0.62646369 \ldots,  0.26555126  0.26411794
  0.0120324 ]
    \end{Verbatim}

    Okay, this looks good. Now lets bin and histogram it using the command
hist from pyplot. Note that I pass it the number of bins I want so that
python will automatically make their width 0.01. Just to be sure, we'll
print the bin sizes (which I call bins here) to check.

    \begin{Verbatim}[commandchars=\\\{\}]
{\color{incolor}In [{\color{incolor}7}]:} \PY{k}{def} \PY{n+nf}{histplot}\PY{p}{(}\PY{n}{data}\PY{p}{)}\PY{p}{:}    
            
            \PY{n}{font} \PY{o}{=} \PY{p}{\PYZob{}}\PY{l+s}{\PYZsq{}}\PY{l+s}{family}\PY{l+s}{\PYZsq{}} \PY{p}{:} \PY{l+s}{\PYZsq{}}\PY{l+s}{monospace}\PY{l+s}{\PYZsq{}}\PY{p}{,}
                    \PY{l+s}{\PYZsq{}}\PY{l+s}{weight}\PY{l+s}{\PYZsq{}} \PY{p}{:} \PY{l+s}{\PYZsq{}}\PY{l+s}{bold}\PY{l+s}{\PYZsq{}}\PY{p}{,}
                    \PY{l+s}{\PYZsq{}}\PY{l+s}{size}\PY{l+s}{\PYZsq{}}   \PY{p}{:} \PY{l+m+mi}{18}\PY{p}{\PYZcb{}}
        
            \PY{n}{matplotlib}\PY{o}{.}\PY{n}{rc}\PY{p}{(}\PY{l+s}{\PYZsq{}}\PY{l+s}{font}\PY{l+s}{\PYZsq{}}\PY{p}{,} \PY{o}{*}\PY{o}{*}\PY{n}{font}\PY{p}{)}
        
            \PY{n}{num\PYZus{}bins} \PY{o}{=} \PY{l+m+mi}{10000}\PY{o}{*}\PY{l+m+mf}{0.01}   \PY{c}{\PYZsh{} Set the number of bins I want.}
        
            \PY{c}{\PYZsh{} Bin the data and plot the histogram. Note this function returns the binned data (measurement), the bins themselves (bins)}
            \PY{c}{\PYZsh{} and the patches that python used to construct the histogram, which we don\PYZsq{}t care about here.}
        
            \PY{n}{plt}\PY{o}{.}\PY{n}{figure}\PY{p}{(}\PY{n}{figsize}\PY{o}{=}\PY{p}{(}\PY{l+m+mi}{10}\PY{p}{,}\PY{l+m+mi}{8}\PY{p}{)}\PY{p}{,} \PY{n}{dpi}\PY{o}{=}\PY{l+m+mi}{80}\PY{p}{)}
            \PY{n}{plt}\PY{o}{.}\PY{n}{title}\PY{p}{(}\PY{l+s}{\PYZsq{}}\PY{l+s}{Histogram of binned numbers}\PY{l+s}{\PYZsq{}}\PY{p}{)}
            \PY{n}{plt}\PY{o}{.}\PY{n}{xlabel}\PY{p}{(}\PY{l+s}{\PYZsq{}}\PY{l+s}{Bin value}\PY{l+s}{\PYZsq{}}\PY{p}{)}
            \PY{n}{plt}\PY{o}{.}\PY{n}{ylabel}\PY{p}{(}\PY{l+s}{\PYZsq{}}\PY{l+s}{Number of numbers in bin}\PY{l+s}{\PYZsq{}}\PY{p}{)}
        
            \PY{p}{[}\PY{n}{measurements}\PY{p}{,} \PY{n}{bins}\PY{p}{,} \PY{n}{patches}\PY{p}{]} \PY{o}{=} \PY{n}{plt}\PY{o}{.}\PY{n}{hist}\PY{p}{(}\PY{n}{data}\PY{p}{,} \PY{n}{bins}\PY{o}{=}\PY{n}{num\PYZus{}bins}\PY{p}{,} \PY{n+nb}{range}\PY{o}{=}\PY{p}{[}\PY{l+m+mf}{0.0}\PY{p}{,} \PY{l+m+mf}{1.0}\PY{p}{]}\PY{p}{)}
        
            \PY{k}{print} \PY{l+s}{\PYZdq{}}\PY{l+s}{The number of bins is}\PY{l+s}{\PYZdq{}}\PY{p}{,} \PY{n}{num\PYZus{}bins}
        
            \PY{k}{print} \PY{l+s}{\PYZdq{}}\PY{l+s}{The bins are:}\PY{l+s}{\PYZdq{}}
        
            \PY{k}{print} \PY{n}{bins}  \PY{c}{\PYZsh{} Check that the bin sizes are correct.}
        
            \PY{n}{plt}\PY{o}{.}\PY{n}{show}\PY{p}{(}\PY{p}{)}
            
            \PY{k}{return} \PY{n}{measurements}\PY{p}{,} \PY{n}{bins}
\end{Verbatim}

    \begin{Verbatim}[commandchars=\\\{\}]
{\color{incolor}In [{\color{incolor}8}]:} \PY{p}{[}\PY{n}{measurements}\PY{p}{,} \PY{n}{bins}\PY{p}{]} \PY{o}{=} \PY{n}{histplot}\PY{p}{(}\PY{n}{rands}\PY{p}{)}
\end{Verbatim}

    \begin{Verbatim}[commandchars=\\\{\}]
The number of bins is 100.0
The bins are:
[ 0.    0.01  0.02  0.03  0.04  0.05  0.06  0.07  0.08  0.09  0.1   0.11
  0.12  0.13  0.14  0.15  0.16  0.17  0.18  0.19  0.2   0.21  0.22  0.23
  0.24  0.25  0.26  0.27  0.28  0.29  0.3   0.31  0.32  0.33  0.34  0.35
  0.36  0.37  0.38  0.39  0.4   0.41  0.42  0.43  0.44  0.45  0.46  0.47
  0.48  0.49  0.5   0.51  0.52  0.53  0.54  0.55  0.56  0.57  0.58  0.59
  0.6   0.61  0.62  0.63  0.64  0.65  0.66  0.67  0.68  0.69  0.7   0.71
  0.72  0.73  0.74  0.75  0.76  0.77  0.78  0.79  0.8   0.81  0.82  0.83
  0.84  0.85  0.86  0.87  0.88  0.89  0.9   0.91  0.92  0.93  0.94  0.95
  0.96  0.97  0.98  0.99  1.  ]
    \end{Verbatim}

    \begin{center}
    \adjustimage{max size={0.9\linewidth}{0.9\paperheight}}{Homework_1_files/Homework_1_7_1.png}
    \end{center}
    { \hspace*{\fill} \\}
    
    Now for the Graduate Student part of this question. We would expect that
the \# in any bin would be 10,000/num\_bins=100. To check this, we can
calculate the \(\chi^2\) of this and see if any deviation from this is
small or not. To calculate \(\chi^2\) we do:

\begin{align} \chi^2 = \sum_i d_i^2/e_i \\ d_i = o_i - e_i \end{align}

where \(e_i\) is each expected measurement, \(o_i\) is the actual
measurement (here what the code returned) and \(d_i\) is the deviation
of the actual observed measures from the expected measures. So doing
this in the code enviroment:

    \begin{Verbatim}[commandchars=\\\{\}]
{\color{incolor}In [{\color{incolor}10}]:} \PY{n}{num\PYZus{}bins} \PY{o}{=} \PY{l+m+mi}{100}
         \PY{n}{expected} \PY{o}{=} \PY{l+m+mi}{10000}\PY{o}{/}\PY{n}{num\PYZus{}bins}
         
         \PY{k}{print} \PY{l+s}{\PYZdq{}}\PY{l+s}{Your expected measurement is}\PY{l+s}{\PYZdq{}}\PY{p}{,} \PY{n}{expected}
         
         \PY{n}{deviations} \PY{o}{=} \PY{n}{measurements} \PY{o}{\PYZhy{}} \PY{n}{expected}
         
         \PY{n}{chi\PYZus{}2} \PY{o}{=} \PY{l+m+mf}{0.0}
         
         \PY{k}{for} \PY{n}{i} \PY{o+ow}{in} \PY{n+nb}{range}\PY{p}{(}\PY{n+nb}{len}\PY{p}{(}\PY{n}{deviations}\PY{p}{)}\PY{p}{)}\PY{p}{:}
             
             \PY{n}{chi\PYZus{}2} \PY{o}{=} \PY{n}{chi\PYZus{}2} \PY{o}{+} \PY{n}{deviations}\PY{p}{[}\PY{n}{i}\PY{p}{]}\PY{o}{*}\PY{o}{*}\PY{l+m+mf}{2.0}\PY{o}{/}\PY{n}{expected}
             
         \PY{k}{print} \PY{l+s}{\PYZdq{}}\PY{l+s}{Your Chi\PYZca{}2 is}\PY{l+s}{\PYZdq{}}\PY{p}{,} \PY{n}{chi\PYZus{}2}
         
         \PY{n}{above90} \PY{o}{=} \PY{p}{(}\PY{n}{chi\PYZus{}2} \PY{o}{\PYZhy{}} \PY{l+m+mf}{117.407}\PY{p}{)}\PY{o}{/}\PY{l+m+mf}{117.407}
         \PY{n}{above90} \PY{o}{=} \PY{l+m+mf}{0.90} \PY{o}{+} \PY{l+m+mf}{0.90}\PY{o}{*}\PY{n}{above90}
         
         \PY{k}{print} \PY{l+s}{\PYZdq{}}\PY{l+s}{Your confidence level that the distribution is uniform is}\PY{l+s}{\PYZdq{}}\PY{p}{,} \PY{n}{above90}
\end{Verbatim}

    \begin{Verbatim}[commandchars=\\\{\}]
Your expected measurement is 100
Your Chi\^{}2 is 93.28
Your confidence level that the distribution is uniform is 0.715051061691
    \end{Verbatim}

    The number of degrees of freedom here is the number of bins, 100, minus
1 for the \(\chi^2\) test. Looking then at the table for a \(\chi^2\)
distribution at NIST
(http://www.itl.nist.gov/div898/handbook/eda/section3/eda3674.htm) I see
for 99 degrees of freedom that this is a uniform distribution with an
upper value of 117.407 for 90\% confidence and 123.225 for 95\%
confidence. 95\% is the ``standard'' assumption about the accuracy of
the measurements, but if you want to be safer even the 90\% confidence
passes the \(\chi^2\) test. The noise is indeed random.


    \subparagraph{Problem \#2 Construct 10,000 iterations of the standard logistic map
\(x'= \lambda x(1-x)\) for \(x \in [0,1]\) with
\(\ 3.7 \le lambda \le 4.0\) and bin them in intervals of 0.01. Plot the
histogram of this.}


    Okay, no problem. I pick \(\lambda=3.85\) to be right in the middle.
First I'll define the function:

    \begin{Verbatim}[commandchars=\\\{\}]
{\color{incolor}In [{\color{incolor}45}]:} \PY{k}{def} \PY{n+nf}{logistic}\PY{p}{(}\PY{n}{x}\PY{p}{,}\PY{n}{lam}\PY{p}{)}\PY{p}{:}
             
             \PY{n}{x1} \PY{o}{=} \PY{n}{lam}\PY{o}{*}\PY{n}{x}\PY{o}{*}\PY{p}{(}\PY{l+m+mi}{1}\PY{o}{\PYZhy{}}\PY{n}{x}\PY{p}{)}
             \PY{k}{return} \PY{n}{x1}
\end{Verbatim}

    Then we will run it on the same random number generator.

    \begin{Verbatim}[commandchars=\\\{\}]
{\color{incolor}In [{\color{incolor}49}]:} \PY{n}{x\PYZus{}prime} \PY{o}{=} \PY{n}{np}\PY{o}{.}\PY{n}{zeros}\PY{p}{(}\PY{l+m+mi}{10000}\PY{p}{)}
         \PY{n}{lamb} \PY{o}{=} \PY{l+m+mf}{3.85}
         
         
         \PY{k}{for} \PY{n}{i} \PY{o+ow}{in} \PY{n+nb}{range}\PY{p}{(}\PY{l+m+mi}{10000}\PY{p}{)}\PY{p}{:}
             
             \PY{n}{x} \PY{o}{=} \PY{n}{foo}\PY{o}{.}\PY{n}{uniform}\PY{p}{(}\PY{l+m+mf}{0.0}\PY{p}{,} \PY{l+m+mf}{1.0} \PY{o}{+} \PY{n}{sys}\PY{o}{.}\PY{n}{float\PYZus{}info}\PY{o}{.}\PY{n}{epsilon}\PY{p}{)}
             \PY{n}{x\PYZus{}prime}\PY{p}{[}\PY{n}{i}\PY{p}{]} \PY{o}{=} \PY{n}{logistic}\PY{p}{(}\PY{n}{x}\PY{p}{,} \PY{n}{lamb}\PY{p}{)}
             
\end{Verbatim}

    And then plot the histogram of the data.

    \begin{Verbatim}[commandchars=\\\{\}]
{\color{incolor}In [{\color{incolor}55}]:} \PY{p}{[}\PY{n}{measurements}\PY{p}{,} \PY{n}{bins}\PY{p}{]} \PY{o}{=} \PY{n}{histplot}\PY{p}{(}\PY{n}{x\PYZus{}prime}\PY{p}{)}
\end{Verbatim}

    \begin{Verbatim}[commandchars=\\\{\}]
The number of bins is 100.0
The bins are:
[ 0.    0.01  0.02  0.03  0.04  0.05  0.06  0.07  0.08  0.09  0.1   0.11
  0.12  0.13  0.14  0.15  0.16  0.17  0.18  0.19  0.2   0.21  0.22  0.23
  0.24  0.25  0.26  0.27  0.28  0.29  0.3   0.31  0.32  0.33  0.34  0.35
  0.36  0.37  0.38  0.39  0.4   0.41  0.42  0.43  0.44  0.45  0.46  0.47
  0.48  0.49  0.5   0.51  0.52  0.53  0.54  0.55  0.56  0.57  0.58  0.59
  0.6   0.61  0.62  0.63  0.64  0.65  0.66  0.67  0.68  0.69  0.7   0.71
  0.72  0.73  0.74  0.75  0.76  0.77  0.78  0.79  0.8   0.81  0.82  0.83
  0.84  0.85  0.86  0.87  0.88  0.89  0.9   0.91  0.92  0.93  0.94  0.95
  0.96  0.97  0.98  0.99  1.  ]
    \end{Verbatim}

    \begin{center}
    \adjustimage{max size={0.9\linewidth}{0.9\paperheight}}{Homework_1_files/Homework_1_17_1.png}
    \end{center}
    { \hspace*{\fill} \\}
    
    Well that doesn't look very uniform. Lets look at the \(\chi^2\) of it.

    \begin{Verbatim}[commandchars=\\\{\}]
{\color{incolor}In [{\color{incolor}57}]:} \PY{n}{expected} \PY{o}{=} \PY{l+m+mi}{10000}\PY{o}{/}\PY{n}{num\PYZus{}bins}
         
         \PY{k}{print} \PY{l+s}{\PYZdq{}}\PY{l+s}{Your expected measurement is}\PY{l+s}{\PYZdq{}}\PY{p}{,} \PY{n}{expected}
         
         \PY{n}{deviations} \PY{o}{=} \PY{n}{measurements} \PY{o}{\PYZhy{}} \PY{n}{expected}
         
         \PY{n}{chi\PYZus{}2} \PY{o}{=} \PY{l+m+mf}{0.0}
         
         \PY{k}{for} \PY{n}{i} \PY{o+ow}{in} \PY{n+nb}{range}\PY{p}{(}\PY{n+nb}{len}\PY{p}{(}\PY{n}{deviations}\PY{p}{)}\PY{p}{)}\PY{p}{:}
             
             \PY{n}{chi\PYZus{}2} \PY{o}{=} \PY{n}{chi\PYZus{}2} \PY{o}{+} \PY{n}{deviations}\PY{p}{[}\PY{n}{i}\PY{p}{]}\PY{o}{*}\PY{o}{*}\PY{l+m+mf}{2.0}\PY{o}{/}\PY{n}{expected}
             
         \PY{k}{print} \PY{l+s}{\PYZdq{}}\PY{l+s}{Your Chi\PYZca{}2 is}\PY{l+s}{\PYZdq{}}\PY{p}{,} \PY{n}{chi\PYZus{}2}
\end{Verbatim}

    \begin{Verbatim}[commandchars=\\\{\}]
Your expected measurement is 100.0
Your Chi\^{}2 is 8054.92
    \end{Verbatim}

    Okay, the table value for a confidence level of 99.9\% is only 148.230.
So that means we have to reject that this is a uniform
distribution\ldots{} its way above that.


    \subparagraph{Problem \#3: Construct the linear relation between the fold map of
\(y'\) and the logistic map \(x'\) from above, and find \(a(\lambda)\),
given that:}


    \begin{align} \\ x'&=\lambda x(1-x) \\ y'&=a-y^2 \\ y&=mx+b \end{align}

    Okay, lets write out \(y'\) and try to form \(x'\) from it with the
linear relation between them:

    \begin{align} \\ y'&=a-y^2 \\ mx'+b &= a - (mx +b)^2 \\ x' &= \frac{a}{m} - \frac{b}{m} - \frac{1}{m}(m^2x^2 + b^2 + 2mxb) \\
&= \frac{a}{m} - \frac{b}{m} - \frac{b^2}{m} - mx\left(x+\frac{2b}{m} \right) \end{align}

    Comparing this to the logistic map, we see the following relations must
be true.

    \begin{align} a &= b + b^2 \\ m &= -2b \\ m &= \lambda \end{align}

    And using these relations we find that \(a(\lambda)\) must be:

\begin{align} a(\lambda) = \frac{\lambda}{2}\left(\frac{\lambda}{2} - 1 \right) \end{align}


    % Add a bibliography block to the postdoc
    
    
    
    \end{document}
